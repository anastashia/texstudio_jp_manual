\begin{longtable}{XX}
  \caption{エディタオブジェクトの一覧}
  \endfirsthead
  \caption{表の続き}
  \endhead
  \hline
  \multicolumn{2}{c}{\textbf{エディタオブジェクト}}\\
  \textbf{コマンド} & \textbf{説明}\\
  \hline
  \texttt{editor.search(searchFor, {[}options{]}, {[}scope{]}, {[}callback{]})}
    & エディタ上で検索を行う。
    \begin{itemize}
    \item
      \texttt{searchFor}は検索対象テキストである。
      文字列(例:``..'')または正規表現(例:\verb+/[.]{2}/+)のいずれかである。
    \item
      \texttt{options}は文字列で、``i''、``g''、``w''の組み合わせである。
      これにより大文字・小文字の区別なし(case-\emph{i}nsensitive)検索か
      グローバル(\emph{g}lobal)検索(最初の一致発見後も続ける)、
      あるいは単語単位での(whole-\emph{w}ord-only)検索かを指定する。
    \item
      \texttt{scope}は検索範囲を制限するカーソルである
      (\texttt{editor.document().cursor}を見よ)。
    \item
      \texttt{callback}は一致ごとに呼び出される関数である。
      一致位置を表すカーソルが最初の引数として渡される。
    \end{itemize}

    \texttt{searchFor}以外の引数は全て省略可能であり、
    順番は変更してもよい(将来の互換性はないかもしれない)。
    関数は見つかった一致したものの数を返す。\\
  \texttt{editor.replace(searchFor, {[}options{]}, {[}scope{]}, {[}replaceWith{]})}
    & エディタ上で検索と置換を行う。
    これは、単純な文字列またはコールバック関数の\texttt{replaceWith}引数を
    別として\texttt{editor.search}のように振る舞う。
    \texttt{replaceWith}が関数の場合、
    その返り値は\texttt{replaceWith}に渡されたカーソルで表される
    一致物を置換するのに用いられる。\\
  \texttt{editor.replaceSelectedText(newText, {[}options{]})}
    & この関数は現在の選択部をnewTextで置換する。
    何も選択されていない場合newTextを挿入する。\newline
    newTextが関数の場合、選択テキストと対応するカーソルを引数としてその関数が呼び出され、戻り値がnewTextとなる。\newline
    全テキスト置換/挿入にこの関数を使用することが推奨される。
    なぜならこれが複数のカーソル/ブロック選択を正しく扱う最も簡単な方法だからだ。
    ただしこの関数はTXS \textgreater{}= 2.8.5でのみ利用できる。\newline

    オプション(options)は次のプロパティを持つオブジェクトである:
    \begin{itemize}
      \item
        \verb+{"noEmpty": true}+ 置換のみ; 選択部が空の場合何も挿入しない
      \item
        \verb+{"onlyEmpty": true}+ カーソル位置に挿入するのみ;
        空でない選択テキストを変更しない
      \item
        \verb+{"append": true}+
        現在の選択部の末尾にnewTextを追加し、古いテキストを削除しない
      \item
        \verb+{"prepend": true}+
        現在の選択部の先頭にnewTextを追加、古いテキストを削除しない
      \item
        \verb+{"macro": true}+
        newTextを通常のマクロテキストとして扱う(例:\verb+%< %>+プレースホルダを挿入)
      \end{itemize}
      例:\newline
      \texttt{editor.replaceSelectedText("world", {"append": true} )}
      ``world''を現在の選択部の末尾に追加\newline
      \texttt{editor.replaceSelectedText(function(s){return s.toUpperCase();})}
      現在の選択部を大文字へ変換\\
  \texttt{editor.undo();} & エディタ上で最後のコマンドを元に戻す\\
  \texttt{editor.redo();} & エディタ上で最後のコマンドをやり直す\\
  \texttt{editor.cut();} & 選択部をクリップボードへ切り取る\\
  \texttt{editor.copy();} & 選択部をクリップボードへコピーする\\
  \texttt{editor.paste();} & クリップボードの内容を貼り付ける\\
  \texttt{editor.selectAll();} & 全てを選択する\\
  \texttt{editor.selectNothing();} & 何も選択しない(選択を解除する)\\
  \texttt{editor.find();} & 「検索パネル」を起動する\\
  \texttt{editor.find(QString text, bool highlight, bool regex, bool word=false, bool caseSensitive=false);}
    & 事前に定義された値で「検索パネル」を起動する\\
  \texttt{editor.find(QString text, bool highlight, bool regex, bool word, bool caseSensitive, bool fromCursor, bool selection);}
    & 事前に定義された値で「検索パネル」を起動する\\
  \texttt{editor.findNext();} & 次を検索する\\
  \texttt{editor.replacePanel();} & (検索パネルが開いていて何か選択物がある場合)置換する\\
  \texttt{editor.gotoLine();} & 「指定行へ移動パネル」を起動する\\
  \texttt{editor.indentSelection();} & 選択部をインデントする\\
  \texttt{editor.unindentSelection();} & 選択部のインデントを解除する\\
  \texttt{editor.commentSelection();} & 選択部をコメント化する\\
  \texttt{editor.uncommentSelection();} & 選択部のコメント化を解除する\\
  \texttt{editor.clearPlaceHolders();} & プレースホルダーを削除する\\
  \texttt{editor.nextPlaceHolder();} & 次のプレースホルダーへ移動する\\
  \texttt{editor.previousPlaceHolder()} & 前のプレースホルダーへ移動する\\
  \texttt{editor.setPlaceHolder(int i, bool selectCursors=true);}
    & プレースホルダーを設定する\\
  \texttt{editor.setFileName(f);} & ファイル名を\emph{f}に設定する\\
  \texttt{editor.write(str)}
    & 現在のカーソル位置に文字列\texttt{str}を挿入する
    (ミラーカーソルがある場合そのすべてに\texttt{str}が挿入される)\\
  \texttt{editor.insertText(str)}
    & 現在のカーソル位置に文字列\texttt{str}を挿入する
    (ミラーカーソルは無視されるので、replaceSelectedTextの使用または代わりに書き込むのに好ましい)\\
  \texttt{editor.setText(\emph{text})}
    & 現在の文書のテキスト全てを\texttt{\emph{text}}で置換する\\
  \texttt{editor.text()} & 完全な文書のテキストを返す\\
  \texttt{editor.text(int line)} & 行\texttt{\emph{line}}のテキストを返す\\
  \hline
\end{longtable}
