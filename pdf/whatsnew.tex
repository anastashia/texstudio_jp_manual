\chapter{変更点}

\section{Version 2.6.6 -\textgreater{} Version 2.7.0}

\begin{itemize}
\item
  「編集」 -\textgreater{} 「テキスト操作」(小文字化 / 大文字化 / タイトルケース化)の追加
\item
  最近使用したセッションのリストを追加
\item
  大きな文書の保存の高速化
\item
  読み込み(import subimport importfrom subimportfrom)に対して
ファイルツリー上で認識されるように改良(Steven Vandekerckhoveのおかげである)
\item
  実行中のコンパイルの停止ボタンを追加
\item
  辞書検索のパスに複数のディレクトリを含めることができるように改良
\item
  OpenOffice/LibreOffice拡張フォーマット(*.oxt)の辞書も利用できるように改良
\item
  LaTeXリファレンスマニュアルを新しいバージョンのものに更新
\item
  画像ツールチップの最大幅に対するオプションを新しく追加
\item
  「コンテキストメニューでの参照コマンド」オプションを新しく追加
\item
  エディタロジックの追加検索パスに対するオプションを新しく追加
\item
  エンコーディングの自動検出に対するオプションの新規追加:
LaTeXベースと文字ベースの検出を別個に選択可能に変更
\item
  PDFの強調表示の色と持続時間に対するオプションを新しく追加
\item
  検索ダイアログ:読み込まれている文書全て、つまり隠れている文書も検索するように変更
\item
  PDF -\textgreater{} ソースの同期の改良
\item
  TeXShopとTeXWorksとの互換性のため``\verb+% !BIB = biber+''構文のサポートを追加
\item
  cwlファイルをいくつか新しく追加
\item
  いくつかのバグを修正
\end{itemize}

\section{Version 2.6.4 -\textgreater{} Version 2.6.6}

\begin{itemize}
\item
  埋め込みPDFビューワーが開いている場合にhome/endキーが正しく働かないバグを修正
\item
  マクロの略語を修正
\item
  エディタ上でログ項目の位置が更新されないバグを修正
\item
  Windows版インストーラーに署名付加
\end{itemize}

\section{Version 2.6.2 -\textgreater{} Version 2.6.4}

\begin{itemize}
\item
  パッケージスキャナー:インストールされたパッケージをTeXシステムに問い合わせ、存在しないパッケージを強調表示する
\item
  パッケージ補完
\item
  組み込みPDFビューワーで基本的な注釈機能をサポート
\item
  描画速度の向上(特にMac)
\item
  隠れた文書の読み込みの高速化(オプション:含まれるファイルの自動読み込み)
\item
  コマンド補完ウィンドウ(特に引用)の高速化
\item
  ログパネルの改良
\item
  dtx強調表示の改良
\item
  LilyPond book (.lytex)のサポートを追加
\item
  「編集 -\textgreater{} 行操作」が選択部でも動作するよう改良
\item
  hunspell libraryを1.3.2へ更新
\item
  修正:インプットメソッドのバグ
\item
  修正:日本語で矢印キーを含むショートカットが機能していなかった問題
\item
  さらなるバグ修正、例:ツールチップがすぐに消えてしまう問題の修正
\end{itemize}

\section{Version 2.6.0 -\textgreater{} Version 2.6.2}

\begin{itemize}
\item
  popplerとWindowsでのQt(4.8.5)を更新
\item
  構造ツリービュー:再帰的に構造を閉じる/展開するためのコンテキストメニュー項目を追加
\item
  結合行での厳密な行のワードラップを改良
\item
  「表示 -\textgreater{}
  ビューワーにフォーカスを移動」が別枠ビューワーに対しても機能するように改良
\item
  LanguageToolと辞書の検出が向上
\item
  「列を揃える」がtabu/longtabuに対しても機能するように改良
\item
  多数のバグを修正:編集可能なユーザーテンプレート、右から左向きに読む言語での\}、Macでのpinyin入力メソッドの問題、……
\end{itemize}

\section{Version 2.5.2 -\textgreater{} Version 2.6}

\begin{itemize}
\item
  パッケージの解説文書PDFが内部PDFビューワーで表示されるように変更
\item
  モダン形式に対するMacでの完全なretinaサポートを実施
\item
  画面全体でより読みやすくするため、組み込みビューワーを拡大/縮小できるように改良
\item
  最後に開いた(そしてまだ閉じていない)環境をalt+enterで閉じることができるように改良
\item
  「列を揃える」がより多くの環境で機能するように改良
\item
  テンプレートリソースがtemplate\_resources.xmlを通して設定し、utf-8で書かれるように変更
\item
  基本的なPweave強調表示を追加
\item
  \textbf{\texttt{\% !TeX spellcheck = ...}}マジックコメントを追加
\item
  現在の構文強調に依存する新しいマクロトリガーを追加
\item
  右から左方向に読む言語に対する双方向表記サポートの改良
\item
  いくつかの小さな修正
\end{itemize}

\section{Version 2.5.1 -\textgreater{} Version 2.5.2}

\begin{itemize}
\item
  任意の領域を折りたたみ可能としてしるし付けするための\verb+%BEGIN_FOLD+
  \ldots{} \verb+%END_FOLD+コメントを新規追加
\item
  PDFビューワーでCJKとキリル文字の表示のサポートを追加
\item
  タブ幅の最大値を32へ増加
\item
  基本的なインプットメソッドのサポートを修正
\item
  LinuxとMac OS Xでのテンプレートの欠損を修正
\item
  Mac OS Xでメニューバーが消える問題を修正
\item
  すでに開いているファイルとして保存する際にクラッシュする点を修正
\item
  長いステータスメッセージのせいでビューワーのサイズが変化しうる問題を修正
\item
  「次/前の文章」に対するショートカットをCtrl+PgDown/Upに変更
\item
  いくつかの小さな修正
\end{itemize}

\section{Version 2.5 -\textgreater{} Version 2.5.1}

\begin{itemize}
\item
  新しいテンプレートシステム
\item
  折りたたみパネルの改良
\item
  エディターとビューワーでの前方向/後方向マウスボタンのサポートを追加
\item
  インラインプレビューのコンテキストメニューの追加(プレビュー画像のコピー可能に)
\item
  参照/コマンドの概要を完全にするためすべての含まれるファイルの読み込みのオプションを追加
\item
  \verb+\bibliography{}+コマンドに対する「開く」コンテキストメニュー項目とリンクの重ねあわせの追加
\item
  図の名前の上に来た時に図のプレビューを表示
\item
  いくつかのバグ修正(PDFのスクロール範囲、ユーザーテンプレートパス、OSX関連のバグ、……)
\end{itemize}

\section{Version 2.4 -\textgreater{} Version 2.5}

\begin{itemize}
\item
  カーソル履歴の追加(後退/前進)
\item
  参照、パッケージ、インクルードされるファイルの名前をCtrl+MouseOver時にリンクになるよう変更
\item
  手書きの数式の挿入機能を追加(Windows 7のみ、TexTablet使用)
\item
  好みの書式を指定するオプションを含む、表コード書式の改良
\item
  LaTeXテンプレートと表テンプレートでのメタデータのサポートを追加
\item
  ランタイムライブラリのバージョンが正しいか確認する機能を追加
\item
  コンテキストメニューをさらに追加(折りたたみパネル、ブックマークパネル)
\item
  より見やすくするためもっと太いカーソルをオプションとして追加
\item
  行操作の追加:上/下へ移動、重複行への操作
\item
  Windowsインストーラー:.texファイルをTXSへ関連付ける選択肢を追加
\item
  いくつかのバグ修正(クラッシュ、コンパイル、焦点移動、……)
\end{itemize}

\section{Version 2.3 -\textgreater{} Version 2.4}

\begin{itemize}
\item
  いくつかのコマンドを容易に組み合わせることができるビルドシステムに刷新
\item
  多数の新ツールのサポート:xelatex, lualatex, biber, latexmk, texindy
\item
  埋め込みPDFビューワーを追加
\item
  ブックマークマネージャと永続的ブックマークを追加
\item
  LanguageToolを用いたインライン文法チェックを追加
\item
  luaとdtxファイルの構文強調表示を追加
\item
  biblatexサポートを追加
\item
  他のアプリケーションから引用を挿入する引用APIを追加(JabRefプラグインを利用可能)
\item
  表の自動整形
\item
  外観の改良
\item
  アップデートチェッカーを追加
\item
  スクリプトの拡張:GUI/ダイアログの作成、他の文章/プログラム/メニューへの接続、バックグラウンドモードとイベント
\item
  クラッシュからの保護機能を追加
\item
  多数のちょっとした改良
\item
  いくつかのバグ修正
\end{itemize}

\section{Version 2.2 -\textgreater{} Version 2.3}

\begin{itemize}
\item
  \verb+\ref+/\verb+\cite+の参照を変更可能なコマンドのリストを追加
\item
  検索履歴の記録機能を追加
\item
  文章ごとに異なる辞書を使用する機能のサポートを追加
\item
  無効な括弧を見つける機能を追加
\item
  ほぼ単語レベルでの逆方向PDF検索機能を追加
\item
  図の挿入マクロでのファイル名の補完機能を追加
\item
  BibTeXの自動呼び出し機能を改良
\item
  スクリプトで利用可能な更に多くの手法を追加
\item
  いくつかのバグ修正(特にPDFビューワー/構文チェック/構造ビューでのクラッシュ)と細かい改良
\end{itemize}

\section{Version 2.1 -\textgreater{} Version 2.2}

\begin{itemize}
\item
  プレビューの改良:

  \begin{itemize}
  \item
    PDFビューワーで複数のページを連続して表示可能に改良
  \item
    PDFビューワーを(マルチスレッドで)非停止で機能するように改良
  \item
    プレビューがインクルードされたファイルで機能するように改良
  \end{itemize}
\item
  任意のユーザーマクロを実行できるようにキーを置換
\item
  ダブルクォートの置換を予め定義しておいたリストから容易に選択できるよう改良
\item
  補完で通常のコマンド、最も頻繁に使用されるもの、すべての選択可能なものの区別をするように改良
\item
  プロファイルの保存/読み込みが機能するように改良
\item
  構文強調される環境を増加
\item
  バグ修正と細かい改良
\end{itemize}

\section{Version 2.0 -\textgreater{} Version 2.1}

\begin{itemize}
\item
  オンラインLaTeXの構文チェックの拡張

  \begin{itemize}
  \item
    表中の列数のチェック
  \item
    文章中でどのコマンドが有効か決めるために\verb+\usepackage+と
    \verb+\documentclass+を利用するように改良
  \item
    新規コマンドの追加
  \end{itemize}
\item
  TXSが読み込んだ文章の親/子関係を自動検出しそれに応じて振る舞うように変更。
  従ってマスターモードはもう必要ない。
\item
  プレビューの改良:

  \begin{itemize}
  \item
    PDFビューワーで複数のページを開けるように改良
  \item
    PDFビューワーでのプレゼンテーションモードと複数ビューのサポートを追加
  \item
    PDFビューワーの外観と時計ドックを更新
  \item
    選択部プレビューの高速化とテキスト中表示への対応
  \end{itemize}
\item
  括弧の選択の容易化
\item
  バグ修正と細かい改良
\end{itemize}

\section{Version 1.9.9a -\textgreater{} Version 2.0}

\begin{itemize}
\item
  PDFビューワーと順方向/逆方向検索を結合
\item
  (単純なエラーに対する)オンラインLaTeXの構文チェックを追加
\item
  表の操作をサポート(行、列、あるいは\verb+\hline+の追加/削除)
\item
  挿入された括弧を自動的に閉じるように変更
\item
  厳密な折り返しを伴う行の長さを制限するオプションを追加
\item
  単語の繰り返しを潜在的なスタイルの間違いとしてしるし付けするように改良
\item
  バグ修正と細かい改良
\end{itemize}

\section{Version 1.9.9 -\textgreater{} Version 1.9.9a}

\begin{itemize}
\item
  Macでのいくつかのパフォーマンス問題への取り組み。Macでの長い行を高速化。
\item
  ひとつ以上の重ね書きを同時に表示可能に改良(例:構文強調とスペルチェック)
\item
  補完されたコマンドのコマンド置換を追加
\item
  切り取りバッファを追加。
  選択されたテキストが補完を通じてコマンドで置換された場合、除去されたテキストが挿入されたコマンドの引数として使用される(適用できる場合)。
\item
  補完でのツールチップで選択された参照の示すラベルの周囲が表示されるように改良
\item
  予め定義しておいたショートカット、メニューの再定義、エディターの設定を含むプロファイルのファイルからの取り込みを追加
\item
  環境名上でテキストカーソルが待機しているときに、
  その環境名を(\verb+\begin+と\verb+\end+同時に)変更できるミラーカーソルを生成するように改良
\item
  ALT-delのタイプで単語やコマンド、環境を削除するように改良
\item
  既知のテキストコマンドでのみスペルチェックを行うように変更
\item
  いくつかのダイアログを小さな画面サイズでもよりうまく対処できるように修正
\item
  ユーザーフィードバック後多数のバグを除去
\end{itemize}

\section{Version 1.9.3 -\textgreater{} Version 1.9.9}

\begin{itemize}
\item
  現在の文書を操作するためのjavaスクリプトがユーザータグで使用可能になる。
  直接のカーソル処理を通じて操作。
  さらなる機能が必要な場合、自由に機能要望を上げてよい。
\item
  Macでのいくつかのパフォーマンス問題への取り組み。
  まだ完全ではないが、Mac上でずっと速く感じるはず。
\item
  式の境界(\$、\verb+begin{equation}+、……)上にマウスを合わせると数式構造物をプレビューできるように変更
\item
  ツールバーをカスタマイズ可能に変更
\item
  Math/LatexメニューでのLaTeXの式をユーザーの好みのバージョンに変更可能に改良
\item
  開いている文書すべてに対する全検索の改良
\item
  SVNを通じたテキスト文書の透過的なバージョン管理のサポート
\item
  構造ビューとカスタムコマンド補完をタイプした時に更新されるように改良
\item
  構造ビューで付録の一部やインクルードされたファイルの欠如のようなところを色付けするように変更
\item
  マスター文書が定義してある場合、開いてある(!)サブ文章から参照とラベルを対話式のラベルチェックと補完に用いることができるように改良
\item
  ddeコマンドを対応するプログラムが起動していない場合に起動するよう改良
\item
  起動したLaTeXがフリーズした場合に2秒後にエスケープキーを押すことで停止できるよう改良
\item
  折りたたみは今や本当に有益である:不一致な括弧があってももはや折りたたみに支障ないし、折りたたまれたブロックを編集することも可能
\item
  ユーザーフィードバック後多数のバグを除去
\end{itemize}

\section{Version 1.8.1 -\textgreater{} Version 1.9}

TeXstudioは前の月の間にこつこつと拡張されてきた。
次のリストは新しく追加された機能や変更されたもののおそらく不完全な概要である:

\begin{itemize}
\item
  動的な構文強調の最初のステップを実装してきた。
  例えば、\verb+\label+や\verb+\ref+のようなコマンド中の参照がチェックされ、
  (参照の場合)その参照が存在しない場合や複数回定義されている場合にはしるし付けられる。
\item
  単語補完システムを拡張してきた。
  今や既知のコマンドを相当数に拡張している``kile''の単語リストを使用している。
  bashシェルのように現在の候補リスト中で共通の単語ベースの補完をTabキーで行える。
  更に、以前に使用されたテキスト部分を提示することで通常のテキストをも補完できる。
  この2つのモードはバックスラッシュが開始文字かどうかで区別される。
  そして最後に、補完プロセス時に置換されたユーザー定義の略語を用いて``ユーザータグ''(ユーザー定義のテキストブロック)を挿入できる。
  キーシーケンスを用いてユーザータグを挿入する以前の方法はまだ利用可能である。
  終わりに、ユーザー定義のLaTeXコマンドは自動的にスキャンされ、コマンド補完で使用出来る。
\item
  ウィザードを使う場合は別として、テンプレートを用いて新しく文章を作成できる。
  ユーザーは必要に応じてあとで編集したり削除したりできる自分のテンプレートを追加できる。
\item
  記号パネルが拡張された。``kile''の記号リストでも拡張された。
  また原文そのままのリストから``タグ''を挿入できる。
  そして最後に、列のカウントが利用可能な水平方向のスペースに自動的に適応する。
  不必要な記号リストを隠せることは言うまでもない。
\item
  記号リストセレクタをより空きを増やすためtexmakerのように左端に移動した。
\item
  カーソルを合わせた際のヘルプを実装した。
  カーソルを標準のLaTeXコマンドの上に合わせるとツールチップヘルプが表示される。
  参照の上に合わせた場合、そのラベルを含む対応するテキストメッセージがツールチップとして表示される。
\item
  選択したテキストのプレビューをステータスパネルかツールチップとして表示するようにした。
\item
  望むならステータス/ログ/エラーパネルをタブで使用出来る。
\item
  今やオンラインスペルチェッカーは\verb+``a+や\verb+\''{a}+のようなエスケープ文字を正しく扱える。
  また、\verb+\ref{label}+などといった(いくつかの)LaTeXコマンドオプションのスペルチェックを控える。
\item
  構造ビューのコンテキストメニューで節全体の選択や節のインデントといった有益なオプションを実行できる。
  これは\verb+\section+を\verb+\subsection+に変更し、
  それに応じてすべての含まれる見出しも変更することができることを意味している。
\item
  類語辞典を追加した。これで単語の一部でも検索できるようになった。
\item
  確かに多数のバグを潰してきた!
\end{itemize}
